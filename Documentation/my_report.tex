\documentclass{report}
\usepackage[hidelinks]{hyperref}

\begin{document}
\tableofcontents
\chapter{Research}
\section{Related Projects}
After gathering requirements from our client we decided to try and find
applications with a similar design or feature set, to decide on a simple way to
build the overall systems architecture and refine the initial requirement that
we got from the client. However, there did not seem to be much on the market
that matched the clients full description i.e.\ a system at allows a normal user
to tag virtual reality videos, with text, html, audio \ldots, and then allows
them to play those tagged videos back on a mobile device such as the Samsung
Gear VR\@. The closest I could find to an app that fulfilled the clients
specification was an application called ``ThingLink'' at
\url{http://demo.thinglink.com/vr-editor} but the ability to edit video comes
out to around £125 per user per month which is prohibitively expensive for
charity work as well as that there is no guaranty that the off the shelf
software can be updated to fit all of the clients needs as requirements change.
But ThingLink did give us the idea to create a Client/Server application that
would export datafiles that any client could use. And in the tradition of other
android file formats we decided the best thing to do would be to use a renamed
zip file with a metadata file inside it.

\chapter{Design and Implementation}

\section{Design}
\subsection{Server Design}

\subsection{Client Design}

\section{Implementation}

\chapter{Compatibility + Response Design Testing}
\end{document}
